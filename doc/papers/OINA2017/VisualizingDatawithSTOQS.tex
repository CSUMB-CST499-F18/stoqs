
%% bare_conf.tex
%% V1.3
%% 2007/01/11
%% by Michael Shell
%% See:
%% http://www.michaelshell.org/
%% for current contact information.
%%
%% This is a skeleton file demonstrating the use of IEEEtran.cls
%% (requires IEEEtran.cls version 1.7 or later) with an IEEE conference paper.
%%
%% Support sites:
%% http://www.michaelshell.org/tex/ieeetran/
%% http://www.ctan.org/tex-archive/macros/latex/contrib/IEEEtran/
%% and
%% http://www.ieee.org/

%%*************************************************************************
%% Legal Notice:
%% This code is offered as-is without any warranty either expressed or
%% implied; without even the implied warranty of MERCHANTABILITY or
%% FITNESS FOR A PARTICULAR PURPOSE! 
%% User assumes all risk.
%% In no event shall IEEE or any contributor to this code be liable for
%% any damages or losses, including, but not limited to, incidental,
%% consequential, or any other damages, resulting from the use or misuse
%% of any information contained here.
%%
%% All comments are the opinions of their respective authors and are not
%% necessarily endorsed by the IEEE.
%%
%% This work is distributed under the LaTeX Project Public License (LPPL)
%% ( http://www.latex-project.org/ ) version 1.3, and may be freely used,
%% distributed and modified. A copy of the LPPL, version 1.3, is included
%% in the base LaTeX documentation of all distributions of LaTeX released
%% 2003/12/01 or later.
%% Retain all contribution notices and credits.
%% ** Modified files should be clearly indicated as such, including  **
%% ** renaming them and changing author support contact information. **
%%
%% File list of work: IEEEtran.cls, IEEEtran_HOWTO.pdf, bare_adv.tex,
%%                    bare_conf.tex, bare_jrnl.tex, bare_jrnl_compsoc.tex
%%*************************************************************************

% *** Authors should verify (and, if needed, correct) their LaTeX system  ***
% *** with the testflow diagnostic prior to trusting their LaTeX platform ***
% *** with production work. IEEE's font choices can trigger bugs that do  ***
% *** not appear when using other class files.                            ***
% The testflow support page is at:
% http://www.michaelshell.org/tex/testflow/



% Note that the a4paper option is mainly intended so that authors in
% countries using A4 can easily print to A4 and see how their papers will
% look in print - the typesetting of the document will not typically be
% affected with changes in paper size (but the bottom and side margins will).
% Use the testflow package mentioned above to verify correct handling of
% both paper sizes by the user's LaTeX system.
%
% Also note that the "draftcls" or "draftclsnofoot", not "draft", option
% should be used if it is desired that the figures are to be displayed in
% draft mode.
%
\documentclass[conference]{IEEEtran}
% Add the compsoc option for Computer Society conferences.
%
% If IEEEtran.cls has not been installed into the LaTeX system files,
% manually specify the path to it like:
% \documentclass[conference]{../sty/IEEEtran}





% Some very useful LaTeX packages include:
% (uncomment the ones you want to load)


% *** MISC UTILITY PACKAGES ***
%
%\usepackage{ifpdf}
% Heiko Oberdiek's ifpdf.sty is very useful if you need conditional
% compilation based on whether the output is pdf or dvi.
% usage:
% \ifpdf
%   % pdf code
% \else
%   % dvi code
% \fi
% The latest version of ifpdf.sty can be obtained from:
% http://www.ctan.org/tex-archive/macros/latex/contrib/oberdiek/
% Also, note that IEEEtran.cls V1.7 and later provides a builtin
% \ifCLASSINFOpdf conditional that works the same way.
% When switching from latex to pdflatex and vice-versa, the compiler may
% have to be run twice to clear warning/error messages.






% *** CITATION PACKAGES ***
%
%\usepackage{cite}
% cite.sty was written by Donald Arseneau
% V1.6 and later of IEEEtran pre-defines the format of the cite.sty package
% \cite{} output to follow that of IEEE. Loading the cite package will
% result in citation numbers being automatically sorted and properly
% "compressed/ranged". e.g., [1], [9], [2], [7], [5], [6] without using
% cite.sty will become [1], [2], [5]--[7], [9] using cite.sty. cite.sty's
% \cite will automatically add leading space, if needed. Use cite.sty's
% noadjust option (cite.sty V3.8 and later) if you want to turn this off.
% cite.sty is already installed on most LaTeX systems. Be sure and use
% version 4.0 (2003-05-27) and later if using hyperref.sty. cite.sty does
% not currently provide for hyperlinked citations.
% The latest version can be obtained at:
% http://www.ctan.org/tex-archive/macros/latex/contrib/cite/
% The documentation is contained in the cite.sty file itself.

\usepackage{hyperref}




% *** GRAPHICS RELATED PACKAGES ***
%
\ifCLASSINFOpdf
  \usepackage[pdftex]{graphicx}
  % declare the path(s) where your graphic files are
  % \graphicspath{{../pdf/}{../jpeg/}}
  % and their extensions so you won't have to specify these with
  % every instance of \includegraphics
  \DeclareGraphicsExtensions{.pdf,.jpeg,.png}
\else
  % or other class option (dvipsone, dvipdf, if not using dvips). graphicx
  % will default to the driver specified in the system graphics.cfg if no
  % driver is specified.
  \usepackage[dvips]{graphicx}
  % declare the path(s) where your graphic files are
  \graphicspath{{../eps/}}
  % and their extensions so you won't have to specify these with
  % every instance of \includegraphics
  \DeclareGraphicsExtensions{.eps}
\fi
% graphicx was written by David Carlisle and Sebastian Rahtz. It is
% required if you want graphics, photos, etc. graphicx.sty is already
% installed on most LaTeX systems. The latest version and documentation can
% be obtained at: 
% http://www.ctan.org/tex-archive/macros/latex/required/graphics/
% Another good source of documentation is "Using Imported Graphics in
% LaTeX2e" by Keith Reckdahl which can be found as epslatex.ps or
% epslatex.pdf at: http://www.ctan.org/tex-archive/info/
%
% latex, and pdflatex in dvi mode, support graphics in encapsulated
% postscript (.eps) format. pdflatex in pdf mode supports graphics
% in .pdf, .jpeg, .png and .mps (metapost) formats. Users should ensure
% that all non-photo figures use a vector format (.eps, .pdf, .mps) and
% not a bitmapped formats (.jpeg, .png). IEEE frowns on bitmapped formats
% which can result in "jaggedy"/blurry rendering of lines and letters as
% well as large increases in file sizes.
%
% You can find documentation about the pdfTeX application at:
% http://www.tug.org/applications/pdftex





% *** MATH PACKAGES ***
%
%\usepackage[cmex10]{amsmath}
% A popular package from the American Mathematical Society that provides
% many useful and powerful commands for dealing with mathematics. If using
% it, be sure to load this package with the cmex10 option to ensure that
% only type 1 fonts will utilized at all point sizes. Without this option,
% it is possible that some math symbols, particularly those within
% footnotes, will be rendered in bitmap form which will result in a
% document that can not be IEEE Xplore compliant!
%
% Also, note that the amsmath package sets \interdisplaylinepenalty to 10000
% thus preventing page breaks from occurring within multiline equations. Use:
%\interdisplaylinepenalty=2500
% after loading amsmath to restore such page breaks as IEEEtran.cls normally
% does. amsmath.sty is already installed on most LaTeX systems. The latest
% version and documentation can be obtained at:
% http://www.ctan.org/tex-archive/macros/latex/required/amslatex/math/





% *** SPECIALIZED LIST PACKAGES ***
%
%\usepackage{algorithmic}
% algorithmic.sty was written by Peter Williams and Rogerio Brito.
% This package provides an algorithmic environment fo describing algorithms.
% You can use the algorithmic environment in-text or within a figure
% environment to provide for a floating algorithm. Do NOT use the algorithm
% floating environment provided by algorithm.sty (by the same authors) or
% algorithm2e.sty (by Christophe Fiorio) as IEEE does not use dedicated
% algorithm float types and packages that provide these will not provide
% correct IEEE style captions. The latest version and documentation of
% algorithmic.sty can be obtained at:
% http://www.ctan.org/tex-archive/macros/latex/contrib/algorithms/
% There is also a support site at:
% http://algorithms.berlios.de/index.html
% Also of interest may be the (relatively newer and more customizable)
% algorithmicx.sty package by Szasz Janos:
% http://www.ctan.org/tex-archive/macros/latex/contrib/algorithmicx/




% *** ALIGNMENT PACKAGES ***
%
%\usepackage{array}
% Frank Mittelbach's and David Carlisle's array.sty patches and improves
% the standard LaTeX2e array and tabular environments to provide better
% appearance and additional user controls. As the default LaTeX2e table
% generation code is lacking to the point of almost being broken with
% respect to the quality of the end results, all users are strongly
% advised to use an enhanced (at the very least that provided by array.sty)
% set of table tools. array.sty is already installed on most systems. The
% latest version and documentation can be obtained at:
% http://www.ctan.org/tex-archive/macros/latex/required/tools/


%\usepackage{mdwmath}
%\usepackage{mdwtab}
% Also highly recommended is Mark Wooding's extremely powerful MDW tools,
% especially mdwmath.sty and mdwtab.sty which are used to format equations
% and tables, respectively. The MDWtools set is already installed on most
% LaTeX systems. The lastest version and documentation is available at:
% http://www.ctan.org/tex-archive/macros/latex/contrib/mdwtools/


% IEEEtran contains the IEEEeqnarray family of commands that can be used to
% generate multiline equations as well as matrices, tables, etc., of high
% quality.


%\usepackage{eqparbox}
% Also of notable interest is Scott Pakin's eqparbox package for creating
% (automatically sized) equal width boxes - aka "natural width parboxes".
% Available at:
% http://www.ctan.org/tex-archive/macros/latex/contrib/eqparbox/





% *** SUBFIGURE PACKAGES ***
%\usepackage[tight,footnotesize]{subfigure}
% subfigure.sty was written by Steven Douglas Cochran. This package makes it
% easy to put subfigures in your figures. e.g., "Figure 1a and 1b". For IEEE
% work, it is a good idea to load it with the tight package option to reduce
% the amount of white space around the subfigures. subfigure.sty is already
% installed on most LaTeX systems. The latest version and documentation can
% be obtained at:
% http://www.ctan.org/tex-archive/obsolete/macros/latex/contrib/subfigure/
% subfigure.sty has been superceeded by subfig.sty.



%\usepackage[caption=false]{caption}
%\usepackage[font=footnotesize]{subfig}
% subfig.sty, also written by Steven Douglas Cochran, is the modern
% replacement for subfigure.sty. However, subfig.sty requires and
% automatically loads Axel Sommerfeldt's caption.sty which will override
% IEEEtran.cls handling of captions and this will result in nonIEEE style
% figure/table captions. To prevent this problem, be sure and preload
% caption.sty with its "caption=false" package option. This is will preserve
% IEEEtran.cls handing of captions. Version 1.3 (2005/06/28) and later 
% (recommended due to many improvements over 1.2) of subfig.sty supports
% the caption=false option directly:
%\usepackage[caption=false,font=footnotesize]{subfig}
%
% The latest version and documentation can be obtained at:
% http://www.ctan.org/tex-archive/macros/latex/contrib/subfig/
% The latest version and documentation of caption.sty can be obtained at:
% http://www.ctan.org/tex-archive/macros/latex/contrib/caption/




% *** FLOAT PACKAGES ***
%
%\usepackage{fixltx2e}
% fixltx2e, the successor to the earlier fix2col.sty, was written by
% Frank Mittelbach and David Carlisle. This package corrects a few problems
% in the LaTeX2e kernel, the most notable of which is that in current
% LaTeX2e releases, the ordering of single and double column floats is not
% guaranteed to be preserved. Thus, an unpatched LaTeX2e can allow a
% single column figure to be placed prior to an earlier double column
% figure. The latest version and documentation can be found at:
% http://www.ctan.org/tex-archive/macros/latex/base/



%\usepackage{stfloats}
% stfloats.sty was written by Sigitas Tolusis. This package gives LaTeX2e
% the ability to do double column floats at the bottom of the page as well
% as the top. (e.g., "\begin{figure*}[!b]" is not normally possible in
% LaTeX2e). It also provides a command:
%\fnbelowfloat
% to enable the placement of footnotes below bottom floats (the standard
% LaTeX2e kernel puts them above bottom floats). This is an invasive package
% which rewrites many portions of the LaTeX2e float routines. It may not work
% with other packages that modify the LaTeX2e float routines. The latest
% version and documentation can be obtained at:
% http://www.ctan.org/tex-archive/macros/latex/contrib/sttools/
% Documentation is contained in the stfloats.sty comments as well as in the
% presfull.pdf file. Do not use the stfloats baselinefloat ability as IEEE
% does not allow \baselineskip to stretch. Authors submitting work to the
% IEEE should note that IEEE rarely uses double column equations and
% that authors should try to avoid such use. Do not be tempted to use the
% cuted.sty or midfloat.sty packages (also by Sigitas Tolusis) as IEEE does
% not format its papers in such ways.





% *** PDF, URL AND HYPERLINK PACKAGES ***
%
%\usepackage{url}
% url.sty was written by Donald Arseneau. It provides better support for
% handling and breaking URLs. url.sty is already installed on most LaTeX
% systems. The latest version can be obtained at:
% http://www.ctan.org/tex-archive/macros/latex/contrib/misc/
% Read the url.sty source comments for usage information. Basically,
% \url{my_url_here}.





% *** Do not adjust lengths that control margins, column widths, etc. ***
% *** Do not use packages that alter fonts (such as pslatex).         ***
% There should be no need to do such things with IEEEtran.cls V1.6 and later.
% (Unless specifically asked to do so by the journal or conference you plan
% to submit to, of course. )


% correct bad hyphenation here
\hyphenation{op-tical net-works semi-conduc-tor}


\usepackage[section]{placeins}
\usepackage{upquote}


\begin{document}
%
% paper title
% can use linebreaks \\ within to get better formatting as desired
\title{Visualizing Data with STOQS, the \\Spatial Temporal Oceanographic Query System}


% author names and affiliations
% use a multiple column layout for up to three different
% affiliations
\author{\IEEEauthorblockN{Mike McCann}
\IEEEauthorblockA{Monterey Bay Aquarium Research Institute \\
Moss Landing, CA USA\\
mccann@mbari.org}
}

% conference papers do not typically use \thanks and this command
% is locked out in conference mode. If really needed, such as for
% the acknowledgment of grants, issue a \IEEEoverridecommandlockouts
% after \documentclass

% for over three affiliations, or if they all won't fit within the width
% of the page, use this alternative format:
% 
%\author{\IEEEauthorblockN{Michael Shell\IEEEauthorrefmark{1},
%Homer Simpson\IEEEauthorrefmark{2},
%James Kirk\IEEEauthorrefmark{3}, 
%Montgomery Scott\IEEEauthorrefmark{3} and
%Eldon Tyrell\IEEEauthorrefmark{4}}
%\IEEEauthorblockA{\IEEEauthorrefmark{1}School of Electrical and Computer Engineering\\
%Georgia Institute of Technology,
%Atlanta, Georgia 30332--0250\\ Email: see http://www.michaelshell.org/contact.html}
%\IEEEauthorblockA{\IEEEauthorrefmark{2}Twentieth Century Fox, Springfield, USA\\
%Email: homer@thesimpsons.com}
%\IEEEauthorblockA{\IEEEauthorrefmark{3}Starfleet Academy, San Francisco, California 96678-2391\\
%Telephone: (800) 555--1212, Fax: (888) 555--1212}
%\IEEEauthorblockA{\IEEEauthorrefmark{4}Tyrell Inc., 123 Replicant Street, Los Angeles, California 90210--4321}}




% use for special paper notices
%\IEEEspecialpapernotice{(Invited Paper)}




% make the title area
\maketitle


\begin{abstract}
%\boldmath

With the ability to collect more data from increasingly more sophisticated robotic platforms, 
the problem of analyzing these data becomes increasing difficult. The Monterey Bay Aquarium 
Research Institute (MBARI) designed the open source Spatial Temporal Oceanographic Query 
System (STOQS) to address this problem.
STOQS uses a geospatial database and a web-based user interface (UI) 
enabling exploration of large collections of data by scientists. The UI is optimized to provide a 
quick overview of data in spatial and temporal dimensions, as well as in parameter and 
platform space. A user may zoom into a feature of interest and select it, initiating a 
filter operation updating the UI with an overview of all the data in the new filtered 
selection. When details are desired, radio buttons and check boxes can be selected to 
generate a number of different types of visualizations. These include color-filled temporal 
section plots, parameter-parameter plots, and both 2D and 3D spatial visualizations.
  

\end{abstract}
% IEEEtran.cls defaults to using nonbold math in the Abstract.
% This preserves the distinction between vectors and scalars. However,
% if the conference you are submitting to favors bold math in the abstract,
% then you can use LaTeX's standard command \boldmath at the very start
% of the abstract to achieve this. Many IEEE journals/conferences frown on
% math in the abstract anyway.

% no keywords




% For peer review papers, you can put extra information on the cover
% page as needed:
% \ifCLASSOPTIONpeerreview
% \begin{center} \bfseries EDICS Category: 3-BBND \end{center}
% \fi
%
% For peerreview papers, this IEEEtran command inserts a page break and
% creates the second title. It will be ignored for other modes.
\IEEEpeerreviewmaketitle


\section{Overview}

The Monterey Bay Aquarium Research Institute (MBARI) designed the Spatial Temporal Oceanographic 
Query System (STOQS) \cite{7054414} to create new capabilities for scientists to gain 
insight from their data. STOQS employs open standards and is a 100\% free and open 
source project. It includes a web-based graphical user interface where the X3D graphics 
standard \cite{x3d07} has been integrated to enable immersive 3D exploration of geospatial data.

MBARI has used STOQS over the last six years to manage and visualize data collected 
during measurement campaigns where scientific goals have centered on improving 
our understanding of biological and geological processes. Fig.~\ref{fig:MUSE_illus_pp} 
is an artist's rendering showing the kinds of platforms and geographic extent for one of
these campaigns.  The data consist primarily of measurements 
collected by moving platforms, but also measurements from stationary platforms. 
The platforms typically have accurate clocks, Global Positioning 
Sensors, underwater inertial navigation sensors, and one or more instruments that measure 
parameters such as temperature, salinity, oxygen, nitrate, chlorophyll fluorescence, 
optical backscatter, particle sizes, and vehicle engineering data. Some platforms also collect
water samples for later laboratory analysis, whose data are also loaded into STOQS. 

The STOQS software is provided for free to the oceanographic community. Its site on
GitHub \cite{stoqs_github} provides simple instructions for building your own server that
you can use to serve your own data.

\begin{figure}[htbp]
\centering
\includegraphics[width=3.3in]{../AUV2014/MUSE_illus_pp.jpg}
\caption{Illustration of an upper water column oceanographic measurement campaign in Monterey Bay, California. 
Data from ships, moorings, gliders, AUVs, ROVs, and benthic instruments are handled by STOQS.}
\label{fig:MUSE_illus_pp}
\end{figure}

\section{STOQS System Architecture}

\subsection{Web stack}

STOQS consists of a PostgreSQL/PostGIS database, Mapserver, and Python-Django 
running on a server and client-side technology (HTML5, CSS, jQuery, OpenLayers, Flot,
X3DOM, Bootstrap) running in a modern web browser (Fig.~\ref{fig:STOQSArch}). 
The web application provides faceted search capabilities allowing a user to quickly 
drill into data of interest. The X3DOM JavaScript library provides interactive 
3D views of the data in browsers that support WebGL.  

\begin{figure}[htbp]
\centering
\includegraphics[width=3.3in]{../AUV2014/stoqs_arch_simple.png}
\caption{Open source components of the server and the web-based client. Programatic access to 
STOQS data is provided through the Python GeoDjango Object Relational Mapping framework 
and through direct SQL to the PostgreSQL database.}
\label{fig:STOQSArch}
\end{figure}

\subsection{CF-NetCDF and OPeNDAP}

Standards are important for the durability of data archived from oceanographic measurement programs. 
These data are unique and costly to collect. One of the standards used within 
the oceanographic community is NetCDF \cite{Rew1990}, a binary data format used 
commonly for gridded numerical model output and remote sensing data. Because of its 
20-plus year history of use, its flexibility, and its recent adoption as an 
Open Geospatial Consortium (OGC) standard, CF-NetCDF and Discrete Sampling Geometries
\cite{DSG} are used to archive MBARI's 
\textit{in situ} measurement data.

\subsection{Relational database rationale}

Measurement trajectory data are
structured in NetCDF files with one coordinate dimension: \texttt{time}. 
  The only index to the data stored in the trajectory format is \texttt{time}. 
Accessing spatial subsets of of these kind of NetCDF files
is inefficient.  For example, 
if an application program needs to extract just the upper 5 meter temperatures from 
a glider data file --- where the glider samples the water column in a sawtooth 
pattern from 0 to 500 meters depth --- it would have to read all 
the data from the file and then select the data less than 5 meters using the program's 
memory space.  This is an
inefficient and memory intensive way to access data; we would rather have the data selection
and subsetting happen during the read from disk. This problem is commonly
encountered by users when they start working with large collections of data 
using commonly available Matlab and Python routines -- they quickly run out of memory. 
STOQS solves this problem by storing the 
data in a geospatial relational database where any field, such as depth 
or geographic location, can be indexed for efficient access --
The open source PostgreSQL 
database along with the PostGIS extension is used. This allows arbitrarily large collections
of data to be efficiently accessed for all sorts of data analysis and visualization
purposes.

\subsection{OPeNDAP}

NetCDF files generated from a campaign are loaded into a STOQS database following an Extract Translate
Load process that depends on the files being served by an OPeNDAP server \cite{Cornillon03opendap:accessing}.
Requiring the data to be served by OPeNDAP before loading into STOQS
supports good data management practice. Knowing the source of data in a data visualization 
system and being able to discover the processing steps all the way back to the original source
of data is important for being able to trust the output of the system. The STOQS User Interface
provides a Metadata section where OPeNDAP Data Access Form links are provided for the sources 
of data that are being viewed.

\subsection{Database design}
An ideal relational database design must consider many aspects of the scientific 
workflow: data ingestion, efficient storage, uniform search and access 
\cite{Bechini:2013:MSS:2425433.2425647}. Design of the STOQS database schema primarily 
focused on solving the problem of efficient access to \textbf{trajectory} data. 
However, care was taken to ensure that other Discrete Sampling Geometry feature types 
\cite{DSG} that are common in oceanography would be supported by the schema.  Other 
feature types common in MBARI's campaigns are \textbf{timeseries} and \textbf{timeseriesProfile} 
which are produced by mooring deployments with instruments at single and multiple depths. 
A fourth feature type, \textbf{trajectoryProfile}, is produced by profiling instruments 
aboard a moving platform; examples include shipboard ADCP (Acoustic Doppler Velocity Profiler)
and AUV (Autonomous Underwater Vehicle) multibeam water column 
measurements. The STOQS schema efficiently handles all of these data feature types.

Review of existing practices \cite{Wright}, \cite{MODB} informed the proper layout 
of the tables, fields and relationships.  Key elements of the STOQS data storage model 
are shown in Fig.~\ref{fig:stoqs_simple_model}. There are 29 other tables in the model 
that hold metatadata, statistics, and simplified representations which are used by 
the user interface to speed data access and visualization. The complete data model 
is available in the project's code repository wiki \cite{stoqs_github}. 

\begin{figure}[htbp]
\centering
\includegraphics[width=1\linewidth]{../AUV2014/stoqs_simple_model.png}
\caption{Key elements of the STOQS database schema where spatial, temporal, and data 
values are stored. The database is normalized for efficient storage of data such 
that data are not repeated in any of the tables. Temporal and spatial data are stored 
separately in the InstantPoint and Measurement/Sample tables. For trajectory and 
timeseries data there is a one-to-one relationship for records in InstantPoint and 
Measurement/Sample. For timeseriesProfile and trajecotryProfile feature type data 
there are multiple Measurement/Sample records for each InstantPoint record. 
Parameter names are stored as records in the Parameter table allowing for new 
parameters without schema modification.}
\label{fig:stoqs_simple_model}
\end{figure}

\subsection{Metadata}

Metadata and its management are key to the success of any data management system. 
STOQS approaches this issue by relying on the community adopted conventions for 
embedding metadata in the NetCDF files. Important conventions include Climate and 
Forecast \cite{CF}, those from the Argo and OceanSITES communities \cite{Pouliquen2006} 
and the Attribute Conventions for Dataset Discovery \cite{ACDD}. Tools upstream 
of STOQS help ensure proper setting of metadata values; for example, OPeNDAP data 
servers include tools that verify against the ISO 19115 Metadata Standard for 
Geographic Data. Regardless of the content of the metadata contained in the NetCDF 
files all of it will be loaded into STOQS. Global attributes and values are stored 
as Resources and attached to the Activity record. Variable attributes that are not 
fields of the Parameter table are also stored as Resources, but are attached to the 
Parameter record. Resource entries are used throughout the STOQS data model to store 
information that cannot be assigned to other parts of the data model.

Another important aspect of metadata is that the barrier for loading and using data in 
STOQS is low, meaning that a minimum amount of metadata is required to properly 
identify the spatial and temporal coordinates, but no requirement is placed on 
consistent naming of variables. For example, the property "sea water temperature" may 
have names like TEMP, temperature, or water\_temp. All of these data can be loaded and 
visualized in the User Interface. Users familiar with the platforms understand the names 
given. The writer of the NetCDF file must pick appropriate names that will be understood 
by users. If the writer knows that a variable is a defined standard parameter then an 
attribute of "standard\_name" can be assigned with a value from the CF Standard Name 
table \cite{CFSN}. This standard\_name is presented in the User Interface and can be 
used to select similar data from multiple platforms.

\section{Using STOQS UI for Data Exploration}

\subsection{STOQS User Interface}

The STOQS User Interface (UI) displays a map of vehicle tracks and station locations.
To the right of the map is a display of 
vehicle depth profiles over time and station measurement 
depth and time coverage (Fig.~\ref{fig:OverviewFirst}). 
The bold blue letter text 
items are each sections that may be expanded revealing more detail. 
The sections in the right column contain lists of items that may be 
selected for filtering, data selection, and plotting. If a Platform is selected for 
filtering then only the information from that platform is shown in the other sections 
of the interface. 
Likewise, if a Parameter is selected then only the Platforms that measured that
Parameter will be shown.
Any selection initiates an instant update of the other items that 
may be selected. With this faceted search capability the user can quickly narrow a 
selection for data of interest. 

\begin{figure}[htbp]
\centering
\includegraphics[width=1.0\linewidth]{OverviewFirst.png}
\caption{STOQS User Interface. On initial load of the campaign at URL 
\url{http://stoqs.mbari.org:8000/stoqs_simz_oct2014} the user is presented 
with an overview of what platforms measured what parameters and graphical views showing 
a  map of the measurement locations (Spatial) and a chart of the time and depth extent (Temporal).
The ``Platform`` and ``Measured Parameter`` sections are initially closed; opening them 
reveals selectors for filtering and plotting.}
\label{fig:OverviewFirst}
\end{figure}

The overall design approach follows the so-called Shneiderman's mantra: "Overview first, 
zoom and filter, then details-on-demand" \cite{Whitney:2012:DIN:2597850}. This contrasts 
with many other oceanographic data portal web sites which often present the user with empty text boxes, 
leaving unfamiliar users at a loss with what to enter. The STOQS UI adheres to 9 
of the 12 elements of interactive dynamics described by Heer and Shneiderman 
\cite{Heer:2012:IDV:2133416.2146416}.


\subsection{Spatial}

\subsubsection{2D}
The default spatial view in the STOQS UI is a 2D OpenLayers generated map. It shows 
simplified vehicle tracks, stations, and sample locations over the selected basemap. A menu, the plus symbol
in the upper right, permits the selection of different basemaps and 
layers. Any new selection of data will trigger an update
of the map, zooming to the extent of the data. If the user has selected a Parameter for plotting
then she may also check a box below the map to plot the data on the map as well.

\subsubsection{3D}
Next to the Map tab in the Spatial section is the 3D tab. It appears 
only for databases in which terrain model(s) have been specified in the data load process. 
The Data section of
the overlay in the 3D scene allows addition of representations of Measurements,
Samples, and Platforms. The Terrain section permits selection of the terrain
models for display in the scene. Fig.~\ref{fig:Spatial_3D} shows nitrate data 
measured by Dorado over the northern part of Monterey Bay in October of 2014. 
Parameter data are delivered to the Spatial 3D display and represented with the same
colormap as used in the Temporal-Depth window.

\begin{figure}[htbp]
\centering
\includegraphics[width=1.0\linewidth]{Spatial_3D.png}
\caption{Viewing nitrate data measured by Dorado in the 3D view of the STOQS UI.
Using the mouse a user can freely navigate the scene to explore the data from 
whatever angle is desired. To see this in your browser visit: \url{http://bit.ly/2gKnnuN}}
\label{fig:Spatial_3D}
\end{figure}

3D graphics in STOQS is made possible by the X3DOM Javascript library \cite{Behr:2010:SAH:1836049.1836077} 
which implements much of the X3D Standard \cite{x3d07}. The declarative approach of X3D
enables non-graphics programming experts to build rich graphics and interactivity 
using rudimentary web development tools -- basically a text editor. Furthermore, X3D's Geospatial component
\cite{Plesch:2015:XGC:2775292.2775315}
makes it easy to represent all manner of geographical data in customized fashions. 
A particular need for upper water column oceanographic data visualization is to
vertically exaggerate the vertical dimension by a factor of 10 or more and to
represent data with negative elevations with respect to the Earth's surface.
Other popular 3D Geospatial data visualization tools, such as Google Earth 
and Cesium, do not provide these capabilities.

A feature implemented in December 2016 is a ``Google Cardboard'' viewing mode
(Fig.~\ref{fig:Cardboard}) where the 3D scene can be viewed on a smart phone providing a Virtual 
Reality experience. Exploring ways to gain insight from data by creating new
3D and VR viewing experiences is an exciting area for future innovation.

\begin{figure}[htbp]
\centering
\includegraphics[width=1\linewidth]{Cardboard.png}
\caption{Google Cardboard viewing mode screen shot from a smart phone. 
Users can have a Virtual Reality experience of data in STOQS.}
\label{fig:Cardboard}
\end{figure}

\subsection{Temporal}

\subsubsection{Depth}
The default temporal display is the Depth tab where the vertical axis of the plot 
is depth. A depth versus time plot is appropriate for \textbf{trajectory} and 
\textbf{timeseriesProfile} data. The plotting space is interpreted as a 
depth section through time. Any X/Y rectangle of the Temporal-Depth 
plot may be selected by a mouse click and drag operation. 
When this is done all other elements of the UI are updated with data 
from the new Time/Depth region.

Data may be plotted in the Temporal-Depth window by selecting a radio button
in the Plot Data column of the Measured Parameters or Sampled Parameters sections.
Initial color limits are set to the 2.5 and 97.5 percentiles of the data distribution,
which gives good results even if there are outlier data values. Other limits and
color maps may be chosen by clicking on the color bar which exposes a Colormaps section in the UI.

\subsubsection{Parameter}
The other way to display temporal data is in the Parameter tab.
This presents time series data recorded by stationary platforms
(trajectory data may also be displayed in the Parameter tab if a nominal
depth value is provided when the data are loaded).
A time range is selectable with a mouse click and drag operation. This 
feature is effective for data exploration as stationary platforms are able 
to detect temporal events such as a rise in chlorophyll or drop in salinity. 
Selecting such an event also shows mobile platform data (in the Depth tab) 
measured at the same time. By viewing selected trajectory data in the Spatial section 
researchers are able to gain a greater understanding of the spatial 
extent of such events.

\subsection{Parameter Values}

The UI has a Parameter Values section that displays histograms of the data values
of the Parameters in the selection (Fig.~\ref{fig:ParameterValues}). 
Distribution of data from each platform are shown in these plots, 
but the plots are also selectors for data. The user may click and drag over
a range of data values in the histogram and that constraint will be added to the selection criteria
for data displayed in other sections of the UI. One way this is used is to visualize
data on density surfaces as is shown in Fig.~\ref{fig:Isopycnal}.

\begin{figure}[htbp]
\centering
\includegraphics[width=1\linewidth]{ParameterValues.png}
\caption{Campaign database at URL \url{http://stoqs.mbari.org:8000/stoqs_canon_april2014}. 
Parameter Values section expanded with the Temporal section collapsed 
showing histograms plots.}
\label{fig:ParameterValues}
\end{figure}


\section{Example Visualizations}

MBARI's public STOQS server at \url{http://stoqs.mbari.org:8000} hosts a few dozen 
campaigns that are available for exploration. Once a particular visualization is
created in the UI it can be shared by clicking on the ``Share this view'' link
at the top of the window. A unique URL will be generated that can be shared via 
email or other messaging tool. The URLs are long so sometimes it's handy to use
URL shortening services that are available on the Internet.

Demonstrations of visualizing data with STOQS are given in the follow screen shots.

Fig.~\ref{fig:Trajectory} demonstrates visualizing \textbf{trajectory} data. Salinity from AUV
Dorado is selected for plotting with data locations for the R/V \textit{Rachel Carson} and the vertical net tows it collected.
Tracks and traces are shown on the map and in the Temporal-Depth plot. The circles indicate 
locations where Dorado Gulper water samples were taken and the black lines represent the vertical net tows.
Salinity from Dorado is plotted using the Haline color map from \cite{cmocean}. A frontal feature is clearly seen in 
the salinity data; having all of this information displayed together helps the biologist 
interpret analyses of the water sample data.

\begin{figure}[htbp]
\centering
\href{http://j.mp/2ggZ3nO}{\includegraphics[width=1\linewidth]{Trajectory.png}}
\caption{Trajectory data visualization from campaign at URL \url{http://stoqs.mbari.org:8000/stoqs_simz_aug2013}.
Salinity from Dorado AUV is shown along with Net Tows and underway tracks from the R/V Rachel Carson.
A specific visualization such as this can be shared with others using the ``Share this view'' link at the
top of the UI. To see these data in your browser visit this shortened version of that link: \url{http://j.mp/2ggZ3nO}.}
\label{fig:Trajectory}
\end{figure}

Fig.~\ref{fig:Timeseries} demonstrates that multiple parameters from multiple \textbf{timeseries} stations
(in this case moorings) can be plotted together in the Temporal Parameter window. If additional Parameters
are checked for plotting in the Measured Parameters section their data will be added to the plot.
If data have different units additional vertical axes will be added to the plot.
To better visualize long time series the Full Screen button button above the plot may be clicked.

\begin{figure}[htbp]
\centering
\href{http://j.mp/2h1KfIX}{\includegraphics[width=1\linewidth]{Timeseries.png}}
\caption{Timeseries data visualization from campaign at URL \url{http://stoqs.mbari.org:8000/stoqs_canon_may2015}.
Measurements of chlorophyll from two stations along the California coast over a period of a week.
The shortened ``Share this view'' link is: \url{http://j.mp/2h1KfIX}.}
\label{fig:Timeseries}
\end{figure}

Fig.~\ref{fig:TimeseriesProfile} is a plot of \textbf{timeseriesProfile} data (hourly northward sea water velocity)
from an ADCP on the M1 mooring in Monterey Bay. The ``balance'' color map from \cite{cmocean} is used along with 
color limits of -20 and 20 cm/s. Starting with the ``share this view'' URL \url{http://j.mp/2gq66cp} a scientist
can use the mooring data to help interpret data from the other platforms. For example, one can open the
Platforms section and click the Dorado and RachelCarson buttons to plot the locations of their data and
visualize it in relation to the Mooring ADCP data.

\begin{figure}[htbp]
\centering
\href{http://j.mp/2gq66cp}{\includegraphics[width=1\linewidth]{TimeseriesProfile.png}}
\caption{Northward water velocity from an ADCP moored in Monterey Bay. A white centered color map and color
limits of -20 and 20 cm/s have been chosen to best represent these data. The Share this view link is: \url{http://j.mp/2gq66cp}}
\label{fig:TimeseriesProfile}
\end{figure}

To demonstrate a more involved data visualization that is possible in the UI, data from the week-long
campaign conducted in San Pedro Bay in April of 2014 
(\url{http://stoqs.mbari.org:8000/stoqs_canon_april2014}) is used.
Fig.~\ref{fig:Isopycnal} shows chlorophyll fluoresence (\texttt{fl700\_uncorr}) from 9 Dorado missions 
plotted between density levels 25.0 and 25.25. A Parameter-Parameter plot is also generated
by checking the ``Show Parameter-Parameter'' checkbox in the Measured Parameters section
and selecting salinity and temperature for the X and Y values. Because these Parameters
have standard names of \texttt{sea\_water\_salinity} and \texttt{sea\_water\_temperature} the STOQS
UI recognizes it as a standard T-S plot and draws the Sigma-T curves.
These sort of visualizations help scientists understand biologically important processes
on neutral density surfaces in the ocean.

\begin{figure*}[htbp]
\centering
\includegraphics[width=1\linewidth]{Isopycnal.png}
\caption{Chlorophyll fluoresence measurement from a week-long set of missions by AUV Dorado 
selected for plotting between isopycnal values of 25.0 and 25.25. Higher values are seen
on the shelf supporting the hypothesis that the shallower water acts as an incubator for
phytoplankton.}
\label{fig:Isopycnal}
\end{figure*}




\subsection{Beyond the STOQS User Interface}

There are many data analysis and visualization tasks that are difficult to
accomplish in a web application. For these cases users of STOQS are advised to 
use the same Python Django interface to the database that the UI
uses. Sometimes these may be long-running analyses that cross
multiple campaigns or for conducting queries that have not been implemented
in the UI.

Several examples of Python scripts that have been written to analyze and
visualize data are in the stoqs/contrib/analysis directory 
in the repository \cite{stoqs_github}. Project Jupyter 
offers a more visual and interactive way to write Python code and interact
with data. Several examples of Jupyter Notebooks that access and visualize
data from STQOS are provided in the stoqs/contrib/notebooks directory.
For example, one Notebook demonstrates performing queries with geospatial
constraints and another demonstrates analyzing Laser Optical
Plankton Counter size class data from the Dorado AUV.

\section{Conclusion}
The STOQS User Interface is a sophisticated yet simple tool for exploring and visualizing 
large collections  of oceanographic measurement data.  
The utility of STOQS has been proven at MBARI both in 
the field, to monitor and ensure successful operations, and
following campaigns for analysis of multidisciplinary data sets.
Instructions for users to build their own development server are provided
on the STOQS GitHub page \cite{stoqs_github}. One may provision a 
complete STOQS data server in about an hour and then start loading their
own data. MBARI plans to continue to enhance STOQS to meet its needs and we
invite others to join the STOQS development and user communities so that it
can meet the needs of the wider oceanographic community.


% conference papers do not normally have an appendix


% use section* for acknowledgement
\section*{Acknowledgements}

Appreciation is given to the numerous people involved in the development, 
deployment, recovery, operation and data handling for all of the platforms 
whose data are presented here. Development of STOQS is supported by the 
David and Lucile Packard Foundation at the Monterey Bay Aquarium Research Institute.




% trigger a \newpage just before the given reference
% number - used to balance the columns on the last page
% adjust value as needed - may need to be readjusted if
% the document is modified later
%\IEEEtriggeratref{8}
% The "triggered" command can be changed if desired:
%\IEEEtriggercmd{\enlargethispage{-5in}}

% references section

% can use a bibliography generated by BibTeX as a .bbl file
% BibTeX documentation can be easily obtained at:
% http://www.ctan.org/tex-archive/biblio/bibtex/contrib/doc/
% The IEEEtran BibTeX style support page is at:
% http://www.michaelshell.org/tex/ieeetran/bibtex/
\bibliographystyle{IEEEtran}
% argument is your BibTeX string definitions and bibliography database(s)
\bibliography{IEEEabrv,./VisualizingDatawithSTOQS}
%
% <OR> manually copy in the resultant .bbl file
% set second argument of \begin to the number of references
% (used to reserve space for the reference number labels box)
%%\begin{thebibliography}{1}
%%
%%\bibitem{IEEEhowto:kopka}
%%H.~Kopka and P.~W. Daly, \emph{A Guide to \LaTeX}, 3rd~ed.\hskip 1em plus
%%  0.5em minus 0.4em\relax Harlow, England: Addison-Wesley, 1999.
%%\end{thebibliography}




% that's all folks
\end{document}


